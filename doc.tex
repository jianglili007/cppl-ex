% $Id: doc.tex,v 1.1 2002-06-05 13:15:55 m Exp $
% (PDF)LaTeX preample {{{
% vim: filetype=tex
% This is a PDFLaTeX-centric header, may not work with plain LaTeX
\documentclass[a4paper,11pt,oneside]{scrartcl}
%\usepackage{babel}
\usepackage[T1]{fontenc}
\usepackage{ae} % Nice letters, also in Postscript and PDF output.
\usepackage{aecompl} % As above.
%\usepackage{pslatex} % Can be used instead of the ae* packages.
\usepackage[latin1]{inputenc}
\usepackage{amsmath}
\usepackage{amstext}
\usepackage{marvosym} % \Smiley, \Frowny etc., nice.
\usepackage{booktabs} % I find 'texdoc booktabs' enlightening.
\usepackage[squaren,cdot,derived,thickqspace]{SIunits}
\usepackage{ifthen} % if, then, else constructs.
\usepackage{xr} % Cros references.
\usepackage[pdftex]{graphicx} % tell graphics about pdf
\usepackage{verbatim}

% Pretty pretty page headers and footers
\usepackage{fancyhdr} % use fancy headers
\pagestyle{fancy} % make fancyhdr take care of it
% or specify explicitly it oneself:
	% \lhead{}
	% \chead{}
	% \rhead{}
	% below is a very ugly hack to make RCS tags appear correctly ...
	% \lfoot{\texttt{\$Id: doc.tex,v 1.1 2002-06-05 13:15:55 m Exp $ $\$}}
	% well, it works :-), but I think it looks bad in the footer ...
	% \lfoot{}
	% \cfoot{\thepage}
	% \rfoot{}
	% \renewcommand{\headrulewidth}{0.4pt}
	% \renewcommand{\footrulewidth}{0.4pt}

% Graphics and colors.
\usepackage[pdftex]{graphicx} % Tell 'graphicx' om PDFTeX.
\usepackage{color} % More colorful docs, se rgb.tex.

% LGrind is a source code pretty printer for LaTeX in the style of vgrind(1)
\usepackage[lineno5]{lgrind} % include code with linenubmber every 5th line
	% \def\CMfont{\large} % these are optional, it looks better in the
	% \def\KWfont{\large} % default sizes I think, though it's not perfect,
	% \def\VRfont{\large} % but it's just too hairy for me fixing it now.
	% \def\BGfont{\large} % Maybe some day.
\usepackage{hyperref} % an indispensable package with lots of options
\hypersetup{ % all the options for hyperref:
	pdftitle={Answers to Exercises in "The C++ Programming Language"},
	pdfauthor={Morten Liebach},
	pdfsubject={Answers to Exercises in Bjarne Stroustup' "The C++ Programming Language"},
	pdfcreator={Morten Liebach},
	pdfproducer={PDFLaTeX},
	pdfkeywords={C++ Bjarne Stroustrup Exercises},
	pdfview={FitBH}, % default PDF `view' for each link
	pdfstartpage={1}, % which page the PDF file is opened on.
  % These next two wrecks pdfLaTeX output totally ...
  %	pdfstartview={FitBH}, % startup page view.
  %	pdfpagescrop={1 1 1 1}, % default PDF crop box for pages. set of 4 numbers.
	pdfpagemode={UseThumbs},
	pageanchor, breaklinks, % break really long links into 2 or more
	pdftex, bookmarks, bookmarksopen, bookmarksnumbered,
	pdfhighlight={/O},
	extension={pdf}, % if 'xr' package is used link to other .pdf docs
	hyperfigures={true},
	backref={true}, % needs a blank line after each \bibitem.
	pagebackref={true}, 
	hyperindex={true}, % makes the text of index entries into hyperlinks.
	colorlinks, % color link text, not a box around them.
	linkcolor={red4}, % color for normal internal links.
	anchorcolor={blue4}, % color for anchor text.
	citecolor={ForestGreen}, % color for bibligraphical citations in text.
	filecolor={cyan4}, % color for URLs which open local files.
	menucolor={red4}, % color for Acrobat menu items.
	pagecolor={red4}, % color for links to other pages.
	urlcolor={blue4}, % color for linked URLs.
}
\usepackage{thumbpdf} % generate thumbnails of pages in PDFTeX (using perl5)
\frenchspacing

\usepackage{makeidx}
%\usepackage{showidx} % Show all \index commands in the margin.
\makeindex

% Definition of a command to insert an 'at-sign', as 'lgrind -e'
% will expand any 'at-sign' found to lgrind macros. Dirty.
\input at % http://kallisti.dk/pub/at.tex
\input rgb % http://kallisti.dk/pub/rgb.tex

% End preample }}}


\title{Answers to the Exercises in \\
``The C++ Programming Language''}
\author{Morten Liebach \\
\texttt{\href{mailto:m\at kallisti.dk}{m\at kallisti.dk}}}

\begin{document}

% Makes producing DVIs easier.
\DeclareGraphicsExtensions{.jpg,.pdf,.mps,.png}

\maketitle

\abstract

\index{The C++ Programming Language}

This is my, very humble, attempt at solving the exercises in Bjarne
Stroustrup' ``The C++ Programming Language''.

Coming from a short introductory course on C++ that tasted like more, I
started working my way through ``The C++ Programming
Language''~\cite{stroustrup}. With no former experience with programming
except the occasional shellscript in my 3 years as a Unix user it has proven
quite hard, but extremely interesting.

If you find errors -- and you will, please don't hesitate to drop me an
email correcting me, I'll be more than happy to incorporate it into this
document with appropriate reference.

\index{BSD license} \index{license}

For the place to get the latest version of this document and the license,
see Appendix \ref{sec:license} page \pageref{sec:license}.

For a complete revision history see Appendix \ref{sec:changelog} page
\pageref{sec:changelog}.

% Here come a lot of stuff ...


\appendix % Appendixes hereafter.

\newpage
\section{License} % {{{
\label{sec:license}

\index{BSD license}

The latest version of this document can be found at
\texttt{\href{http://kallisti.dk/pub/cppl-ex.pdf}
{http://kallisti.dk/pub/cppl-ex.pdf}}, also as \LaTeX sourcecode at
\texttt{\href{http://kallisti.dk/pub/cppl-ex.tex}
{http://kallisti.dk/pub/cppl-ex.tex}}. This document is published under the
following license and Copyright:

\small{
\begin{verbatim}
Copyright (c) 2002 Morten Liebach.  All rights reserved.
 
Redistribution and use in source and binary forms, with or without
modification, are permitted provided that the following conditions
are met:

1. Redistributions of source code must retain the above copyright
   notice, this list of conditions and the following disclaimer.
2. Redistributions in binary form must reproduce the above copyright
   notice, this list of conditions and the following disclaimer in the
   documentation and/or other materials provided with the distribution.
3. All advertising materials mentioning features or use of this software
   must display the following acknowledgement:
     This product includes software written by Morten Liebach.
4. The name of the author may not be used to endorse or promote products
   derived from this software without specific prior written permission.
 
THIS SOFTWARE IS PROVIDED BY THE AUTHOR ``AS IS'' AND ANY EXPRESS OR
IMPLIED WARRANTIES, INCLUDING, BUT NOT LIMITED TO, THE IMPLIED WARRANTIES
OF MERCHANTABILITY AND FITNESS FOR A PARTICULAR PURPOSE ARE DISCLAIMED.
IN NO EVENT SHALL THE AUTHOR BE LIABLE FOR ANY DIRECT, INDIRECT,
INCIDENTAL, SPECIAL, EXEMPLARY, OR CONSEQUENTIAL DAMAGES (INCLUDING, BUT
NOT LIMITED TO, PROCUREMENT OF SUBSTITUTE GOODS OR SERVICES; LOSS OF USE,
DATA, OR PROFITS; OR BUSINESS INTERRUPTION) HOWEVER CAUSED AND ON ANY
THEORY OF LIABILITY, WHETHER IN CONTRACT, STRICT LIABILITY, OR TORT
(INCLUDING NEGLIGENCE OR OTHERWISE) ARISING IN ANY WAY OUT OF THE USE OF
THIS SOFTWARE, EVEN IF ADVISED OF THE POSSIBILITY OF SUCH DAMAGE.
\end{verbatim}
}

I can be reached at
\texttt{\href{mailto:m\at kallisti.dk}{m\at kallisti.dk}}.
%, or by normal mail at:

% Maybe not.
% \begin{quote}
% Morten Liebach \\
% c/o Jens Waring \& Charlotte Waring \\ % Or is it !Waring for CHarlotte?
% Brog\aa{}rdsv\ae{}nget 1 \\
% 2820 Gentofte \\
% Denmark
% \end{quote}

% }}}

\newpage
\section{Changelog} % {{{
\label{sec:changelog}

\begin{flushleft} % Don't indent this. Strictly speaking not necessary.
This version is:
\end{flushleft}

\begin{verbatim}
$Id: doc.tex,v 1.1 2002-06-05 13:15:55 m Exp $
\end{verbatim}

\begin{flushleft} % Don't indent this.
The complete log from CVS:
\end{flushleft}

\begin{verbatim}
$Log: not supported by cvs2svn $
\end{verbatim}

% }}}

% And then:

% Bibliography. Remember to run bibtex(1).
\newpage
\bibliography{cppl-ex}
\bibliographystyle{plain}

% Print the index. Remember to run makeindex(1).
\printindex

\end{document}

